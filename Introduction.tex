	\section{Introduction}
		
		
		\subsection{Sustainable Development}
		
		
		
		\subsection{Development Disparities}
		
		
		
		\subsection{Big Data}
			\subsubsection{Big Data Analyses}
			
			\subsubsection{Big Data for Sustainability}
			
			\subsubsection{title}
			
			
			
		\subsection{Image Classification}
			\subsubsection{Deep Neural Networks}
			
			\subsubsection{YOLO \& Darkflow}
			
			\subsubsection{title}
			
			\subsubsection{title}
			
			
			
		\subsection{Goals of this Study}
		
		Show potentials of big data in combination with machine learning for indicators of SDGs.
		
		
		
			\subsubsection{Research Questions}
			
			In the following paragraph, research questions based on the goals of this study are formulated. Research questions 1 and 1.1 are directly linked to target indicator \#58 of the SDGs. Research question 2 is oriented towards the potential, overall contribution of Big Data for Sustainability.
			
			\bigbreak
			

				\noindent\fbox{%
					\parbox{\textwidth}{%
						\begin{center}
							\textbf{Research Question 1:} \smallbreak Can georeferenced data for indicator \#58 of the SDGs be generated using a Deep Neural Network on the Twitter Streaming API?
						\end{center}		
					}%
				}
				
			\bigbreak
				
				\noindent\fbox{%
					\parbox{\textwidth}{%
						\begin{center}
							\textbf{Research Question 1.1:} \smallbreak Are these data comparable to conventional data for indicator \#58 of the SDGs in terms of quality and accuracy?
						\end{center}			
					}%
				}
				
			\bigbreak
				
				
				\noindent\fbox{%
					\parbox{\textwidth}{%
						\begin{center}
							\textbf{Research Question 2:} \smallbreak What are potentials and limitations of Big Data analyses for the monitoring of the SDGs?
						\end{center}		
					}%
				}
				
			\bigbreak
				
	
			