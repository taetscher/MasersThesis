	
	\section{Introduction}
		
		\subsection{Sustainable Development}
    		This section introduces the concept of Sustainable Development (SD). After a brief overview of its origins important steps in the adoption of SD into global politics by the United Nations are highlighted before various key models are introduced. Finally some of the current measures used to globally advance efforts of SD are presented.\bigskip
		
		    \subsubsection{Origins of Sustainable Development}
		        Ulrich \citet{grober2007} describes how today's notion of Sustainable Development originated from the concept of Sustainability. Grober further elaborates on how the term "Sustainability" was first introduced to the domain of forestry through Hanns-Carl von Carlowitz \citet{voncarlowitz1732} with his \textit{magnum opus} "Sylvicultura Oeconomica" in which he described the necessity of a controlled and sustained use of timber. Timber was an essential resource at the time, that could not be substituted. According to Grober (\citeyear{grober2007}:18), von Carlowitz criticized "the contemporary short-termed way of thinking which was centred solely on making money", thus emphasizing that society should assure a steady supply of timber through conservation and reforestation efforts in order to guarantee the continual and sustained use of the resource.
		        \medskip
		        
		        The following centuries saw authors like Thomas Robert Malthus (\citeyear{malthus1926}) and George Perkins Marsh (\citeyear{marsh1965}) as well as the Club of Rome (\citeyear{meadows1972a}) publish concerns about human overpopulation, resource shortages and a possible system collapse of the world as it was. In The Limits to growth, Donella Meadows and the Club of Rome (\citeyear{meadows1972a}:23) concluded that "if the present growth trends in world population, industrialization, pollution, food production, and resource depletion continue unchanged, the limits to growth on this planet will be reached sometime within the next one hundred years. The most probable result will be a rather sudden and uncontrollable decline in both population and industrial capacity". Jacobus A. du Pisani (\citeyear{dupisani2006}) gives a comprehensive and detailed overview of this period in the history of the idea of Sustainable Development and the various theories on development and progress that preceded it.
		        \medskip
		        
		        More than 200 years would pass after Carlowitz' concerns until the modern notion of Sustainable Development was introduced formally to global politics. Michael \citet{redclift2005} explains that through the report on global environment and development by the \textit{Brundtland Commission}, or "World Commission on Environment and Development" (\citeyear{wced1987}), the term "Sustainable Development" was introduced into political vocabulary. Gro Harlem Brundtland (\citeyear{brundtland1987}:292),  who headed the commission, defines Sustainable Development as development that meets "[...] the needs and aspirations of the present generation without compromising the ability of future generations to meet their needs".
		        
		        
		    
		    \subsubsection{Sustainability as a Geopolitical Paradigm}
		        With the Brundtland definition of Sustainable Development 
		        
		        MDGs & SDGs
		        
		    
		    \subsubsection{Agenda 2030}
		        SDG Agenda 2030, Indicators, Indicator 58
		
		
		\subsection{Development Disparities}
		
		
		
		\subsection{Big Data}
			\subsubsection{Big Data Analyses}
			
			\subsubsection{Big Data for Sustainability}
			
			\subsubsection{title}
			
			
			
		\subsection{Image Classification}
			\subsubsection{Deep Neural Networks}
			
			\subsubsection{YOLO \& Darkflow}
			
			\subsubsection{title}
			
			\subsubsection{title}
			
			
			
		\subsection{Goals of this Study}
		
		Show potentials of big data in combination with machine learning for indicators of SDGs.
		
		
		
			\subsubsection{Research Questions}
			
			In this section, research questions based on the goals of this study are formulated. Research questions 1 and 1.1 are directly linked to target indicator \#58 of the SDGs. Research question 2 is oriented towards the potential overall contribution of Big Data for Sustainability.
			
			\bigskip
			
			\begin{tcolorbox}
				\textbf{Research Question 1:} \smallskip Can georeferenced data for indicator \#58 of the SDGs be generated using a Deep Neural Network on the Twitter Streaming API?
			\end{tcolorbox}
			
			\bigskip
				
				\begin{tcolorbox}
					\textbf{Research Question 1.1:} \smallskip Are these data comparable to conventional data for indicator \#58 of the SDGs in terms of quality and accuracy?
				\end{tcolorbox}
				
			\bigskip
				
				\begin{tcolorbox}
					\textbf{Research Question 2:} \smallskip What are potentials and limitations of Big Data analyses for the monitoring of the SDGs?
				\end{tcolorbox}

			\bigskip
				
	
			